\chapter{Feasible Potential Computation}\label{sec:pot_gen}
Notice that in neither of our proposed \markovs, the engineered check for negative weights requires the weight update itself to be randomized.
In this direction, one could also propose a deterministic sequence of weight updates for a given graph to achieve a desired weight function if no negative cycles are created.
Even better, as our improved \markovs such as \algsp, \algbp, and \algns also compute a feasible potential function $\pot$ for the final weight function $w$, we can utilize these methods to compute a feasible potential function $\pot$ for a deterministic $w$ if possible or output that a negative cycle exists.

As briefly highlighted in \cref{sec:related_work}, this is the standard method for theoretical state-of-the-art \sssp algorithms with negative edge-weights: first compute a feasible potential function $\pot$ or output that a negative cycle exists in which case \sssp is infeasible.
Then run \algdk for weight function $\wpot$ which adheres to optimal runtime bounds to compute the shortest path tree.\footnote{
  One could then subtract the potential gradient to get correct distances with respect to $w$.
}

Hence, we propose a deterministic variant of \algns that makes use of \cref{lem:oblivious_neighbors} to compute a feasible potential function $\pot$ for a given weight function $w$ on graph $G = (V,E)$.

\begin{theorem}\label{thm:det_pot_computation}
  For a given graph $G = (V,E)$ and weight function $w\colon E \to \sR$, let $n^-$ denote the number of nodes in $V$ with incoming negative-weighted edges. 
  Then there is an algorithm that computes a feasible potential function $\pot$ in time \[
    \Oh{n^- \cdot (m + n\log n)}
  \] or outputs \negcycle if there is a negative cycle in the graph.
\end{theorem}
\begin{proof}
  We adapt \algns by starting with weight function \[
    w_+\colon E \to \sRpos, \qquad e \mapsto \max\set{0, w(e)}. 
  \] 
  As $w_+(e) \geq 0$ for every $e \in E$, $\pot = 0$ is feasible and $(G,w_+)$ are consistent.
  We then iteratively re-insert negative edge-weights into $w_+$ to eventually yield $w$ (or abort earlier).
  We can do this one node at a time where we set the weight of all incoming edges of this node to their original weight (\ie resetting the weight of negative-weighted edges) and checking for consistency using the method of \algns.
  This also includes the updates to $\pot$ that are needed to maintain feasibility throughout the algorithm.

  Let $w_+^{(i)}$ denote the (tentative) weight function after $i$ steps of this algorithm.
  Then, by previous observations in \cref{obs:weight_increase_possible,thm:node_sampler_is_uniform}, if $(G,w)$ are consistent, so are $(G,w_+^{(i)})$ as $w_+^{(i)}(e) \geq w(e)$ for every $e \in E$.
  Correctness then follows from the correctness of \algns whereas runtime follows by definition of $n^-$ (as we only need to update incoming edges of nodes that have negative-weighted incoming edges) as well as \cref{lem:dijkstra_runtime}.
\end{proof}

\begin{algorithm}[!htb]
  \caption{
    \algen~\cite{NegSSSPNearLinear,NegSSSPNowFaster}
  }
  \label{algo:elimneg}

  \KwInput{graph $G = (V, E)$, weights $w\colon E \rightarrow \sR$, helper node $s$}
  \KwOutput{$\spt(s) = \pot$ or \negcycle}

  \BlankLine
  \For{$v \in V$}{
    $D[v] \gets \infty$\;
    $P[v] \gets \textsc{Null}$\;
  }
  $D[s] \gets 0$\;
  \BlankLine
  $Q \gets \text{Min-}\textsc{PriorityQueue}$\;
  $S \gets \textsc{Stack}$\;
  add $s$ to $Q$\;
  \Repeat{}{
    \tcp{Dijkstra Phase}
    \While{$Q$ is not empty}{
      $v \gets$ node in $Q$ with minimum $D[v]$\;
      $S.\texttt{push}(v)$\;
      \For{$(v, x) \in E \setminus E^{neg}$\label{line:neg_edges_set}}{
	\If{$D[v] + w(v, x) < D[x]$}{
	  $D[x] \gets D[v] + w(v, x)$\;
	  $P[x] \gets v$\;
	  \lIf{$x \notin Q$}{
	    add $x$ to $Q$
	  }
	}
      }
    }
    \BlankLine
    \tcp{Bellman-Ford Phase}
    \While{$S$ is not empty}{
      $v \gets S.\texttt{pop}()$\;
      \For{$(v, x) \in E$}{ 
	\If{$D[v] + w(v, x) < D[x]$}{
	  $D[x] \gets D[v] + w(v, x)$\;
	  $P[x] \gets v$\;
	  \lIf{$x \notin Q$}{
	    add $x$ to $Q$
	  }
	}
      }
    }
    \BlankLine
    \tcp{Correct Distances}
    \If{$Q$ is empty}{
      \KwRet{$D$}\;
    }

    \BlankLine
    \tcp{Check for Acyclicity of SPT}
    \If{last check is $\Th{n}$ edge-relaxations ago \textbf{or} this is iteration $i > n^-$}{
      \If{$P$ is not acyclic\label{line:acyclicity_check}}{ 	  
	\KwRet{\negcycle}\;
      }
    }
  }
\end{algorithm}

\bigskip

Although \cref{thm:det_pot_computation} is a neat result that shows a more broad way of utilizing our engineered dynamic consistency checks, it is probably of not much practical use in this exact context.
Common methods for potential computation in a practical context include the method of running one \algbf iteration as done so in \algjs (see \cref{sec:preliminaries}) or more recent methods such as \algen~\cite{NegSSSPNearLinear,NegSSSPNowFaster} that alternate between \algbf and \algdk (see \cref{algo:elimneg}).
$E^{neg}$ in \cref{line:neg_edges_set} refers to the set of edges with negative weight with respect to $w$.
Similar to \algjs's method, we start with an additional vertex $s$ that has a zero-weighted edge to every other node in the graph and compute the shortest path tree $\spt(s)$ that by \cref{lem:shortest_paths_are_feasible} is a feasible potential function.

As a personal addition, we also include the amortized acyclicity check used in the SPFA heuristic of \algbf to prevent an endless loop.
Here, we interpret $P[v] = u$ as an edge $(u,v)$ in a tentative shortest path tree (or no edge if $P[v] = \textsc{Null}$): if this ever creates a cycle, this must correspond to a negative cycle in the original graph.
We call this subgraph the $P$-induced subgraph.
Again, this checks runs in time $\Oh{n}$~\cite{toposort} since the tentative shortest path tree has $\Oh{n}$ edges.

Although \cite{NegSSSPNearLinear} assumes constant degree for all nodes, \ie $m = \Th{n}$, in their runtime proof, we can upper bound their runtime for general graphs.
\begin{lemma}[Runtime \algen]\label{lem:runtime_elimneg}
  \algen correctly returns $\spt(s)$ or \negcycle in time \[
    \Oh{n^- \cdot (m + n\log n)}.
  \]
\end{lemma}
\begin{proof}
  We only show general runtime and correctness of the additional negative cycle check as correctness of the original algorithm is done in~\cite{NegSSSPNearLinear}.
  The original authors also prove an upper bound of \[
    \Oh{\log n \cdot \left(n + \sum_{v \in V}\eta(v)\right)}
    \] where \[
      \eta(v) \Def \begin{cases}
	\infty & \text{, if $d(s,v) = -\infty$} \\
	\min\setc{\abs{E^{neg} \cap P}}{\text{$P$ is a shortest \path{s}{v} in $G$}} & \text{otherwise,}
      \end{cases}
    \] \ie the minimum number of negative edges on any shortest \path{s}{v}.

    We can mostly disregard their original analysis and only focus on \cite[Lemma~A.3]{NegSSSPNearLinear} which states that \qq{after iteration $i$ of the \texttt{Dijkstra Phase}, $D[v] = d(s,v)$ for every $v$ where $\eta(v) \leq i$} if no negative cycle is present in the graph (otherwise $D[v] \leq d(s,v)$).
    Then, $v$ can not be added again to $Q$ in the \texttt{Bellman-Ford Phase} as the minimum distance has been found.
    Thus, the runtime of the algorithm for general graphs is upper bounded by the number of iterations times the runtime of both phases combined.
    The \texttt{Dijkstra Phase} trivially runs in $\Oh{m + n\log n}$ and the \texttt{Bellman-Ford Phase} in time $\Oh{m + n}$ as we only iterate over each node and its neighbors at most once.
    Again, by \cite[Lemma~A.3]{NegSSSPNearLinear} it follows that the total runtime is therefore upper bounded by \[
      \Oh{(m + n\log n) \cdot \max_{v \in V}\eta(v)} = \Oh{n^- \cdot (m + n\log n)},
    \] since after $\max_{v \in V}\eta(v)$ iterations, every node has found its correct shortest path and $Q$ is empty.
    Also, trivially $\max_{v \in V}\eta(v) \leq n^-$ because if no negative cycles exist, every shortest path is simple and thus only uses every node at most once.
    The number of negative edges along a shortest path is therefore upper bounded by the number of nodes with incoming negative-weighted edges which is $n^-$.

    In the case where a negative cycle exists, notice that the original algorithm never terminates.
    Therefore, we must at some point reach iteration $n^- + 1$ in which there is at least one node $v$ with $D[v] < d'(s,v)$ where $d'(s,v)$ is the length of the shortest simple \path{s}{v}.
    Otherwise, the algorithm would have already terminated and by the original analysis correctly returned a shortest path tree.
    Then, since $D[v] < d'(s,v)$, the shortest \path{s}{v} must have traversed one node (of a negative cycle) at least twice which implies that the $P$-induced subgraph can not be acyclic.
    Also by definition, the $P$-induced subgraph (shortest path \qq{tree}) can only be acyclic if a negative cycle is present in the graph.
    Thus, the additional check is correct and incurs no worse asymptotic runtime due to amortization.

    Lastly, the latest check that would eventually return \negcycle happens in $\Oh{n^-}$ iterations which also does not incur a worse asymptotic runtime.
\end{proof}

Initial experiments indicated that the proposed method of \cref{thm:det_pot_computation} is by far the worst technique among the discussed three in practice for common graph models.
Thus, at this time, no further research was conducted on this specific method (or similar bounds on \algjs's method).
